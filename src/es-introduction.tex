%
% %CopyrightBegin%
%
% Copyright Ericsson AB 2017. All Rights Reserved.
%
% Licensed under the Apache License, Version 2.0 (the "License");
% you may not use this file except in compliance with the License.
% You may obtain a copy of the License at
%
%     http://www.apache.org/licenses/LICENSE-2.0
%
% Unless required by applicable law or agreed to in writing, software
% distributed under the License is distributed on an "AS IS" BASIS,
% WITHOUT WARRANTIES OR CONDITIONS OF ANY KIND, either express or implied.
% See the License for the specific language governing permissions and
% limitations under the License.
%
% %CopyrightEnd%
%

\chapter{Introduction}

\ifStd
\index{Erlang@\Erlang!\Std|(}
This document is a specification of the language \StdErlang, which will
be available from Ericsson Software Technology AB as \NewErlang. There is a
companion document \cite{olderlang}, which is a specification of the
closely related \Erlang\ implementation
called \OldErlang, developed at Ericsson Software Technology AB.
\index{Erlang@\Erlang!\Std|)}
\else
\index{Erlang@\Erlang!\OldVsn|(}
This document is a specification of the \Erlang\ implementation called
\ErlVsn{4.7.3}, developed at Ericsson Telecom AB.
\index{Erlang@\Erlang!\OldVsn|)}
\fi

\Erlang\ was originally designed by Joe Armstrong\index{Armstrong, Joe},
Robert Virding\index{Virding, Robert}, Claes
Wikstr\"{o}m\index{Wikstr\"{o}m, Claes} and Mike
Wil\-liams\index{Williams, Mike} at the Computer Science Laboratory of
Ericsson Telecommunications Systems Laboratories.

This specification is primarily designed to be useful for \Erlang\
programmers and implementors of \Erlang\ by providing clear although
mostly informally expressed semantics for all language constructs.
It should be of use also for those developing analysis tools for
\Erlang\ although for these purposes the semantics may have to be
reformulated as a formal system.

The specification should thus be able to function well as a reference
manual for the language.  As such it should be a good companion to the
book \emph{Concurrent Programming in ERLANG, Second Edition}
\cite{erlbook}, which is more of a tutorial, text book and evangel
than a language reference.  That edition of the book, however, uses an
earlier version of the language (the implementation \ErlVsn{4.3}).

It is also intended that the specification should be useful as a basis for
a future international standardization of the language.
