%
% %CopyrightBegin%
%
% Copyright Ericsson AB 2017. All Rights Reserved.
%
% Licensed under the Apache License, Version 2.0 (the "License");
% you may not use this file except in compliance with the License.
% You may obtain a copy of the License at
%
%     http://www.apache.org/licenses/LICENSE-2.0
%
% Unless required by applicable law or agreed to in writing, software
% distributed under the License is distributed on an "AS IS" BASIS,
% WITHOUT WARRANTIES OR CONDITIONS OF ANY KIND, either express or implied.
% See the License for the specific language governing permissions and
% limitations under the License.
%
% %CopyrightEnd%
%

\chapter{The external term format}

\label{chapter:external-format}
\index{term!external format|(}

The external term format is a representation of any \Erlang\ term
\ifStd (although possibly subject to tighter implementation limits than elsewhere) \fi
as a sequence of bytes.  It is used as part of the \Erlang\
distribution protocol but can also be accessed explicitly through the
BIFs
\ifOld \T{binary_to_term/1}\index{binary_to_term/1 BIF@\T{binary_to_term/1} BIF}
(\S\ref{section:binarytoterm1}) \fi
\ifStd \T{binary:to_term/1}\index{binary:to_term/1 BIF@\T{binary:to_term/1} BIF}
(\S\ref{section:binary:toterm1}) \fi
and
\ifOld \T{term_to_binary/1}\index{term_to_binary/1 BIF@\T{term_to_binary/1} BIF}
(\S\ref{section:termtobinary1}),\fi
\ifStd \T{term:to_binary/1}\index{term:to_binary/1 BIF@\T{term:to_binary/1} BIF}
(\S\ref{section:term:tobinary1}),\fi
in which a sequence of bytes is represented as
a binary.

The version of the external term format described here is 4.7.  It can
be recognized by the sequence of bytes beginning with a byte \T{131}.
Future versions of the external term format that are not compatible
with version 4.7 must begin the sequence of bytes differently.

We will describe the external term format as a function
$\I{TermRep}_{4.7}$ that given an \Erlang\ term returns a sequence of
bytes.  We use a mathematical notation rather than \Erlang\ function
declarations, although it would not be difficult to transform our
function definitions to an \Erlang\ program that returns a list of
bytes.

\section{Context}

\label{section:atom-tables}

The transformation from a term to a sequence of bytes is context-dependent.
We assume that we can access the name of the node on which a term \T{t}
resides.

\index{node!atom table|(}
\ifOld
There is also an \emph{atom table} that has 256 rows with keys
$0$ to $255$, each of which has an \Erlang\ atom as value.
\fi
\ifStd
There is also an \emph{atom table} with keys $0$ to
$\mathit{atom_table_size}-1$ (where $\mathit{atom_table_size}$ is an
implementation-defined parameter), each of which has an \Erlang\ atom
as value.
\fi
Its purpose is to reduce the communication when terms are being
transmitted between two nodes (although it would be possible to use an
atom table also, for example, when writing terms to a file).  The idea
is that every node has one atom table for each of its
friends\index{node!friendship} (\S\ref{chapter:nodes}).  Two nodes
that are friends, say $\TZ{N}_1$ and $\TZ{N}_2$, agree to ensure that
$\TZ{N}_1$'s atom table for $\TZ{N}_2$ (i.e.,
\T{atom_tables[$\Z{N}_1$]($\Z{N}_2$)}\index{atom_tables node property@\T{atom_tables} node property},
cf.\ \S\ref{section:node-state-dynamic})
will have the same contents as
$\TZ{N}_2$'s atom table for
$\TZ{N}_1$ (i.e., \T{atom_tables[$\Z{N}_2$]($\Z{N}_1$)}).

When an atom \TZ{A} is to be transmitted from node $\TZ{N}_1$ to
node $\TZ{N}_2$, the following
happens at node $\TZ{N}_1$.  First a hash value \TZ{I} between 0 and
\ifOld 255 \fi
\ifStd $\mathit{atom_table_size}-1$ \fi
is computed for \TZ{A} (cf.\ \S\ref{chapter:hashing}).  Then row
\TZ{I} of $\TZ{N}_1$'s atom table for $\TZ{N}_2$ is inspected.  If
that row has \TZ{A} as value, then only \TZ{I} is transmitted to
$\TZ{N}_2$ because it is assumed that also $\TZ{N}_2$'s atom table for
$\TZ{N}_1$ has \TZ{A} as value for \TZ{I}.  Otherwise row \TZ{I} of
$\TZ{N}_1$'s atom table for $\TZ{N}_2$ is updated to contain \TZ{A}
and the whole printname of \TZ{A} is transmitted together with \TZ{I}
to $\TZ{N}_2$.  Note that sequences of bytes must be transformed back
into terms at node $\TZ{N}_2$ in the same order as they were produced
at node $\TZ{N}_1$, for the atom tables to have the correct contents
at all times.
\index{node!atom table|)}

\section{Definitions}

We will use the following auxiliary functions.  When we write that
something is an error, it means that the transformation should fail.

\begin{itemize}
\item $\I{BE}(k,i)$ (for Big Endian) is the sequence of bytes
$\TZ{b}_1$, \ldots, $\TZ{b}_k$ such that
$\I{BigEndianValue}(\langle\TZ{b}_1, \ldots, \TZ{b}_k\rangle)$ is $i$.
It is an error if $i\notin[0,256^k-1]$.
\item $\I{BES}(k,i)$ (for Big Endian Signed) is the sequence of bytes
$\TZ{b}_1$, \ldots, $\TZ{b}_k$ such that $\I{BigEndianSignedValue}(\langle\TZ{b}_1, \ldots, \TZ{b}_k\rangle)$ is $i$.
It is an error if $i\notin[0,256^k-1]$.
\item $\I{ID}(\TZ{t})$, where \TZ{t} is a ref or a port, is the value of
\T{ID[\Z{t}]}, which uniquely identifies the ref or port on the node on
which it resides.
\ifOld
If \TZ{t} is a PID, $\I{ID}(\TZ{t})$ is the XXX least significant bits
of \T{ID[\Z{t}]}, cf.\ $\I{PS}(\TZ{t})$.
\fi
\ifStd
If \TZ{t} is a PID, $\I{ID}(\TZ{t})$ is the $32$ least significant
bits of \T{ID[\Z{t}]}, cf.\ $\I{PS}(\TZ{t})$.
\fi
\item $\I{FloatString}(f)$ is a sequence of 31 bytes where the first 26 (if $f$ is
nonnegative) or 27 (if $f$ is negative) are the character codes of the
string produced by the ISO C \T{printf} facility given the format
string \T{"\char`\%.20e"} and the remaining 4 or 5 bytes are \T{0}.
\item $\I{CR}(\TZ{t})$, where \TZ{t} is a ref, PID or port, is the value of
\T{creation[\Z{t}]} (a nonnegative integer), which distinguishes
between different invocations of nodes with the same name.
$\I{CR}(\TZ{t})$ must be in the range $[0,255]$.
\item $\I{LE}(k,i)$ (for Little Endian) is the unique sequence of bytes
$\TZ{b}_1$, \ldots, $\TZ{b}_k$ such that $\I{LittleEndianValue}(\langle\TZ{b}_1, \ldots, \TZ{b}_k\rangle)$
is $i$.  It is an error if $i\notin[0,256^k-1]$.
\item $\I{Log}(k,i)$ is the base $k$ logarithm of $i$ rounded towards zero.
It is an error if $i\leq 0$.
\item $\I{Node}(t)$ is an atom that names the node on which the term $t$ resides.
\ifOld
\item $\I{PS}(\TZ{t})$, where \TZ{t} is a PID is \T{ID[\Z{t}]} with the XXX least
significant bits removed, cf.\ $\I{ID}(\TZ{t})$.
\fi
\ifStd
\item $\I{PS}(\TZ{t})$, where \TZ{t} is a PID is \T{ID[\Z{t}]} with the $32$ least
significant bits removed, cf.\ $\I{ID}(\TZ{t})$.
\fi
%\item $\I{PrintLength}(a)$ is the length of the printname of the atom $a$.
%\item $\I{PrintName}(a)$ is a sequence of bytes that constitute the character codes
%of the printname of the atom $a$.  It is an error if the printname of $a$ contains
%characters with codes greater than 255.
\item $SignBit(i)$ is \T{0} if $i\geq 0$ and \T{1} if $i<0$.
%\item $Size(t)$ where $t$ is a tuple or binary is the number of elements it has.
\end{itemize}

We write a sequence of bytes $b_1$, \ldots, $b_k$ as $\langle
b_1,\ldots,b_k\rangle$.  Juxtaposition denotes concatenation, so
\[\langle b_1,\ldots,b_k\rangle
\langle b'_1,\ldots,b'_l\rangle = \langle b_1,\ldots,b_k,b'_1,\ldots,b'_l\rangle.\]

\section{The transformation}

\[\I{TermRep}_{4.7}(\TZ{T}) = \langle\T{131}\rangle\,\I{TR}_{4.7}(\TZ{T})\]

The value of $\I{TR}_{4.7}(\TZ{T})$ is defined by cases.  When cases
overlap, either case could be used but the more specific case is
preferred.
\begin{itemize}
\item If \TZ{T} is an integer in the range $[0,255]$, then
\[\I{TR}_{4.7}(\TZ{T}) = \langle\T{97}\rangle\,\langle\TZ{T}\rangle.\]
\item If \TZ{T} is an integer in the range $[-2^{31},2^{31}-1]$
(but typically not in the range $[0,255]$), then
\[\I{TR}_{4.7}(\TZ{T}) = \langle\T{98}\rangle\,\I{BES}(4,\TZ{T}).\]
\item If \TZ{T} is an integer in the range $[-(256^{255}-1),256^{255}-1]$
(but typically not in the range $[-2^{31},2^{31}-1]$), then
\[\I{TR}_{4.7}(\TZ{T}) =
\langle\T{110},l,\I{SignBit}(t)\rangle\,\I{LittleEndian}(l,\TZ{T}),\]
where $l=\I{Log}(256,|\TZ{T}|)+1$.
\item If \TZ{T} is an integer in the range $[-(256^{2^{32}-1}-1),256^{2^{32}-1}-1]$
(but typically not in the range $[-(256^{255}-1),256^{255}-1]$), then
\[\I{TR}_{4.7}(\TZ{T}) =
\langle\T{111}\rangle\,\I{BE}(4,l)\,
\langle\I{SignBit}(\TZ{T})\rangle\,
\I{LittleEndian}(l,|\TZ{T}|),\]
where $l=\I{Log}(256,|\TZ{T}|)+1$.
\item If \TZ{T} is a float, then
\[\I{TR}_{4.7}(\TZ{T}) = \langle\T{99}\rangle,\I{FloatString}(\TZ{T}).\]
\item If \TZ{T} is an atom where the printname consists of $k$ characters with codes
$i_1$, \ldots, $i_k$, such that $k\leq255$ and for all $j$, $1\leq j\leq k$,
$i_1\in[0,255]$, then
\[\I{TR}_{4.7}(\TZ{T}) = \langle\T{100},\T{0},k,i_1,\ldots,i_k\rangle,\]
or if the atom is also to be stored in row $p$ of the atom table (where $p\in[0,255]$),
then
\[\I{TR}_{4.7}(\TZ{T}) = \langle\T{78},p,\T{0},k,i_1,\ldots,i_k\rangle,\]
or if the atom is already stored in row $p$ of the atom table (where $p\in[0,255]$),
then
\[\I{TR}_{4.7}(\TZ{T}) = \langle\T{67},p\rangle.\]
\item If \TZ{T} is a ref, then
\[\I{TR}_{4.7}(\TZ{T}) = \langle\T{101}\rangle\,\I{TR}_{4.7}(\I{Node}(\TZ{T}))\,
\I{BE}(4,\I{ID}(\TZ{T}))\,\langle\I{CR}(\TZ{T})\rangle.\]
\item If \TZ{T} is a port, then
\[\I{TR}_{4.7}(\TZ{T}) = \langle\T{102}\rangle\,\I{TR}_{4.7}(\I{Node}(\TZ{T}))\,
\I{BE}(4,\I{ID}(\TZ{T})),\langle\I{CR}(\TZ{T})\rangle.\]
\item If \TZ{T} is a PID, then
\[\I{TR}_{4.7}(\TZ{T}) = \langle\T{103}\rangle\,\I{TR}_{4.7}(\I{Node}(\TZ{T}))\,
\I{BE}(4,\I{ID}(\TZ{T}))\,\I{BE}(4,\I{PS}(\TZ{T}))
\langle\I{CR}(\TZ{T})\rangle.\]
\item If \TZ{T}is a tuple with elements $\TZ{T}_1$, \ldots, $\TZ{T}_k$, where $k\leq255$, then
\[\I{TR}_{4.7}(\TZ{T}) = \langle\T{104},k\rangle\,
\I{TR}_{4.7}(\TZ{T}_1)\,\cdots\,\I{TR}_{4.7}(\TZ{T}_k).\]
\item If \TZ{T} is a tuple with elements $\TZ{T}_1$, \ldots, $\TZ{T}_k$, where $k\leq2^{32}-1$, then
\[\I{TR}_{4.7}(\TZ{T}) = \langle\T{104}\rangle\,\I{BE}(4,k)\,
\I{TR}_{4.7}(\TZ{T}_1)\,\cdots\,\I{TR}_{4.7}(\TZ{T}_k).\]
\item If \TZ{T} is \T{[]}, then
\[\I{TR}_{4.7}(\TZ{T}) = \langle\T{106}\rangle.\]
\item If \TZ{T} is a (typically nonempty) string of Latin-1 characters having codes
$i_1$, \ldots, $i_k$, where $k\leq2^{16}-1$, then
\[\I{TR}_{4.7}(\TZ{T}) = \langle\T{107}\rangle\,\I{BE}(2,k)\,
\langle i_1,\ldots,i_k\rangle.\]
\item If \TZ{T} is a term \T{[$\TZ{T}_1$,\ldots,$\TZ{T}_k$|$\TZ{T}_{k+1}$]}, where
$k\leq2^{32}-1$, (but typically not a string covered by the previous case) then
\[\I{TR}_{4.7}(\TZ{T}) = \langle\T{108}\rangle\,\I{BE}(4,k)\,
\I{TR}_{4.7}(\TZ{T}_1)\,\cdots\,\I{TR}_{4.7}(\TZ{T}_{k+1}).\]
\item If \TZ{T} is a binary with elements $i_1$, \ldots, $i_k$, where $k\leq2^{32}-1$, then
\[\I{TR}_{4.7}(\TZ{T}) = \langle\T{109}\rangle\,\I{BE}(4,k)\,
\langle i_1,\ldots,i_k\rangle.\]
\item Otherwise, it is an error.
\end{itemize}

The inverse transformation is obvious.
\index{term!external format|)}
