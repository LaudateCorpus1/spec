%
% %CopyrightBegin%
%
% Copyright Ericsson AB 2017. All Rights Reserved.
%
% Licensed under the Apache License, Version 2.0 (the "License");
% you may not use this file except in compliance with the License.
% You may obtain a copy of the License at
%
%     http://www.apache.org/licenses/LICENSE-2.0
%
% Unless required by applicable law or agreed to in writing, software
% distributed under the License is distributed on an "AS IS" BASIS,
% WITHOUT WARRANTIES OR CONDITIONS OF ANY KIND, either express or implied.
% See the License for the specific language governing permissions and
% limitations under the License.
%
% %CopyrightEnd%
%

\chapter{Compiling a module}

\label{chapter:compilation}
\index{module!compilation|(}
The unit of compilation in \Erlang\ is a module.
The compilation of a module is carried out in the following steps:
\begin{enumerate}
\item Lexical processing, as described in \S\ref{chapter:lexical}.
This takes a sequence of
\ifNew Unicode characters (\S\ref{section:unicode}) \fi
\ifOld ASCII characters \fi
and produces a sequence
of tokens and full stops, i.e., a sequence of \NT{TokenSequence}
(\S\ref{section:tokenseq}).
\item Preprocessing, as described in \S\ref{section:preprocessing}.
This takes a sequence of \NT{TokenSequence} and produces a new
sequence of \NT{TokenSequence}, after conditional compilation
and macro expansion.
\item Parsing, as described in \S\ref{section:parsing}.
Each \NT{TokenSequence} in the sequence is parsed as an
\Erlang\ form (\S\ref{chapter:programs-modules}).
The result is a list of parse trees, which are represented as
\Erlang\ terms
(\S\ref{chapter:parse-trees}).
\item Parse transformation, as described in \S\ref{section:parse-transform}.
If a parse transform function has been specified, it is applied
to the list of parse trees,
resulting in a new list of parse trees.  Otherwise the result is the
same list of parse trees.
\item Code generation, as described in \S\ref{section:code-generation}.
This takes a list of parse trees and returns a binary representation
of a module, which can be loaded (\S\ref{chapter:module-dynamics}).
\end{enumerate}
The resulting binary can be loaded into a running \Erlang\ system or be saved in a
file and loaded from there (\S\ref{section:loading}).
\index{module!compilation|)}

\section{Preprocessing}

\label{section:preprocessing}
\index{module!preprocessing of|(}

The preprocessing step takes a sequence of \NT{TokenSequence}
as input and produces a new sequence of \NT{TokenSequence}.
(A \NT{TokenSequence} is a sequence of tokens followed by a full stop,
cf.\ \S\ref{section:tokenseq}.)

\index{skipping mode|(}
\index{processing mode|(}
During preprocessing, the compiler is in one of two modes: skipping
or processing.  When the compiler is in skipping mode, a \NT{TokenSequence}
is ignored unless it is one of the directives for conditional compilation
discussed in \S\ref{section:condcomp}.  When the compiler is in processing mode,
the treatment of a \NT{TokenSequence} is as described in this section.
The compiler is originally in processing mode.
\index{skipping mode|)}
\index{processing mode|)}

Each \NT{TokenSequence} that is one of the following directives
is processed and is not part of the output sequence of \NT{TokenSequence}.
\begin{rules}
\grrule{Directive}
       {\NT{MacroDefinition} & (\S\ref{section:macrodef}) \OR
        \NT{MacroUndefinition} & (\S\ref{section:macroundef}) \OR
        \NT{IncludeDirective} & (\S\ref{section:include}) \OR
        \NT{IncludeLibDirective} & (\S\ref{section:include}) \OR
        \NT{IfdefDirective} & (\S\ref{section:condcomp}) \OR
        \NT{IfndefDirective} & (\S\ref{section:condcomp}) \OR
        \NT{ElseDirective} & (\S\ref{section:condcomp}) \OR
        \NT{EndifDirective} & (\S\ref{section:condcomp})}
\end{rules}
\index{define directive@\T{define} directive|(}
\index{undef directive@\T{undef} directive|(}
The directives \T{-define(\Z{M}$[\text{($\NT{V}_1$,\tdots,$\NT{V}_k$)}]$,\Z{Toks})}
and \T{-undef(\Z{M})} maintain the set of macro definitions,
\index{define directive@\T{define} directive|)}
\index{undef directive@\T{undef} directive|)}
\index{include directive@\T{include} directive|(}
\index{include_lib directive@\T{include_lib} directive|(}
the directives \T{-include(\Z{F})} and \T{-include_lib(\Z{F})} control file inclusion,
\index{include directive@\T{include} directive|)}
\index{include_lib directive@\T{include_lib} directive|)}
\index{ifdef directive@\T{ifdef} directive|(}
\index{ifndef directive@\T{ifndef} directive|(}
\index{else directive@\T{else} directive|(}
\index{endif directive@\T{endif} directive|(}
and the directives \T{-ifdef(\Z{M})}, \T{-ifndef(\Z{M})}, \T{-else} and \T{-endif} control
conditional compilation (\S\ref{section:condcomp}).
\index{ifdef directive@\T{ifdef} directive|)}
\index{ifndef directive@\T{ifndef} directive|)}
\index{else directive@\T{else} directive|)}
\index{endif directive@\T{endif} directive|)}

The tokens in a \NT{TokenSequence} that is not a \NT{Directive} are
subject to macro expansion\index{macro!expansion}
(\S\ref{section:macroexp}) and the resulting tokens, followed by a
full stop, form a \NT{TokenSequence} that is part of the output of the
preprocessing.

\section{Macros}

\label{section:macros}

As the preprocessor goes through the sequence of \NT{TokenSequence},
it maintains a set of macro definitions.  A \NT{MacroDefinition} adds
to the set (\S\ref{section:macrodef}), a \NT{MacroUndefinition} may
remove from the set (\S\ref{section:macroundef}), any other
\NT{TokenSequence} leaves it unchanged.

A \NT{TokenSequence} that is not recognized as a \NT{Directive}
(\S\ref{section:preprocessing}) is subject to macro expansion
(\S\ref{section:macroexp}).

The initial set of macro definitions is described in \S\ref{section:initial-macros}.

\subsection{Macro definition}

\label{section:macrodef}
\index{define directive@\T{define} directive|(}
\index{macro!definition|(}

A macro definition is a directive that adds a macro definition.

\begin{rules}
\grrule{MacroDefinition}
       {\TXT{-} \TXT{define} \TXT{(} \NT{MacroName} \OPT{MacroParams}
       \TXT{,} \NT{MacroBody} \TXT{)} \NT{FullStop}}

\grrule{MacroName}
      {\NT{AtomLiteral} \OR
       \NT{Variable}}

\grrule{MacroParams}
       {\TXT{(} \OPT{Variables} \TXT{)}}

\grrule{Variables}
       {\NT{Variable} \OR
        \NT{Variables} \TXT{,} \NT{Variable}}

\grrule{MacroBody}
       {\NT{Tokens}}
\end{rules}

A macro definition \T{-define(\Z{M},\Z{Toks})} associates the macro
name \TZ{M} with no sequence of parameters and an arbitrary (and
possibly empty) sequence of tokens \TZ{Toks}.

A macro definition
\T{-define(\Z{M}($\NT{V}_1$,\tdots,$\NT{V}_k$),\Z{Toks})}, where
$k\geq0$, associates the macro name \TZ{M} with a (possibly empty)
sequence of macro parameters $\NT{V}_1$, \ldots, $\NT{V}_k$ which must
be distinct variables, and a (possibly empty) sequence of tokens
\TZ{Toks}.

Unlike the case for function names, which associate a symbol and an
arity with a function, a macro name is simply a symbol and the macro
named \TZ{M} either takes no parameters at all or obtains the arity
$k$ above.  It is thus not possible to define two macros with the same
name \TZ{M} but different arities.

In either case, the scope of the association for \TZ{M} begins
immediately after the macro definition.  It is a compile-time error if
a macro definition of \TZ{M} occurs in the scope of another macro
definition of \TZ{M}.

Note the difference between the macro definitions \T{-define(abc,123)}
and \T{-define(abc(),123)}.  The former is associated with no
parameters while the latter is associated with an empty sequence of
parameters and an application of it must therefore have an empty
argument sequence (\S\ref{section:macroexp}).

\ifOld
In \OldErlang, a macro name \T{'Foo'} and a macro name \T{Foo} denote
different macros.  This is to be viewed as a bug (not a feature) and
will change in a future version so a
\fi
\ifStd
A
\fi
macro name that is a quoted atom and a macro name that is a variable
denote the same macro if the print name of the atom consists of the
same sequence of characters as the variable.  For example, the macro
names \T{'Foo'} and \T{Foo} \ifOld will then \fi denote the same
macro.

\index{macro!definition|)}
\index{define directive@\T{define} directive|)}

\subsection{Macro undefinition}

\label{section:macroundef}
\index{macro!undefinition|(}
\index{undef directive@\T{undef} directive|(}

The scope of a macro definition is terminated by a macro undefinition or the end of the
module definition.
\begin{rules}
\grrule{MacroUndefinition}
       {\TXT{-} \TXT{undef} \TXT{(} \NT{MacroName} \TXT{)} \NT{FullStop}}
\end{rules}
Beginning immediately following an undefinition \T{-undef(\Z{M})},
the macro name \TZ{M} is undefined.
It is not an error if \TZ{M} was undefined also immediately preceding the undefinition.
\index{macro!undefinition|)}
\index{undef directive@\T{undef} directive|)}

\subsection{Macro expansion}

\label{section:macroexp}
\index{macro!expansion|(}

Each macro application in a \NT{TokenSequence} is expanded, i.e., replaced with
tokens according to the macro definition that is in force at that point.

\begin{rules}
\grrule{MacroApplication}
       {\TXT{?} \NT{MacroName} \OR
	\TXT{?} \NT{MacroName} \TXT{(} \OPT{MacroArguments} \TXT{)}}

\grrule{MacroArguments}
       {\NT{MacroArgument} \OR
	\NT{MacroArguments} \TXT{,} {MacroArgument}}

\grrule{MacroArgument}
       {\NT{BalancedExpr} that is not one of \TXT{,} or \TXT{)}}

\grrule{BalancedExpr}
       {\TXT{(} \NT{BalancedExprs} \TXT{)} \OR
        \TXT{[} \NT{BalancedExprs} \TXT{]} \OR
        \TXT{\{} \NT{BalancedExprs} \TXT{\}} \OR
        \TXT{begin} \NT{BalancedExprs} \TXT{end} \OR
        \ifNew\TXT{all_true} \NT{BalancedExprs} \TXT{end} \OR
        \TXT{some_true} \NT{BalancedExprs} \TXT{end} \OR
        \TXT{cond} \NT{BalancedExprs} \TXT{end} \OR\fi
        \TXT{if} \NT{BalancedExprs} \TXT{end} \OR
        \TXT{case} \NT{BalancedExprs} \TXT{end} \OR
        \TXT{receive} \NT{BalancedExprs} \TXT{end} \OR
        \ifNew\TXT{try} \NT{BalancedExprs} \TXT{end} \OR\fi
        \ifOld\TXT{query} \NT{BalancedExprs} \TXT{end} \OR\fi
	\NT{OtherToken} \OR
	\NT{BalancedExpr} \NT{BalancedExpr}}
\end{rules}

Macro expansion of a \NT{TokenSequence} is defined as follows.  Let us
write $\langle\TZ{T}_1\ \TZ{T}_2\ \ldots\ \TZ{T}_k\rangle$ for the
macro expansion of a \NT{TokenSequence} $\TZ{T}_1\ \TZ{T}_2\ \ldots\
\TZ{T}_k$.  We write $\tau$ or $\tau'$ for an arbitrary sequence of
tokens and $\beta_1$, \ldots, $\beta_k$ for $k$ balanced expressions,
i.e., \NT{BalancedExpr} above.
\begin{itemize}
\item $\langle\text{\T{?}\ \TZ{A} $\tau$ \NT{FullStop}}\rangle = \langle\text{$\tau'$ $\tau$
\NT{FullStop}}\rangle$\ if \TZ{A} is an atom and there is a macro definition for \TZ{A} with no
parameters and a replacement sequence $\tau'$.
\item $\langle\text{\T{?}\ \TZ{A} \T{(} $\beta_1$ \T{,}\ \ldots\ \T{,}\ $\beta_k$ \T{)} $\tau$ \NT{FullStop}}\rangle = \text{}$ \\
\hfil$\langle\text{$\tau'$[$\beta_1$/$v_1$,\ldots,$\beta_1$/$v_k$] $\tau$ \NT{FullStop}}\rangle$ \\
if \TZ{A} is an atom and there is a macro definition for \TZ{A} with $k$ parameters
$v_1$, \ldots, $v_k$ and a replacement sequence $\tau'$.
\item $\langle\text{\T{?}\ $\tau$ \NT{FullStop}}\rangle$\ in any other case is a compile-time error.
\item $\langle\text{\TZ{T} $\tau$ \NT{FullStop}}\rangle = \text{\TZ{T} $\langle\text{$\tau$ \NT{FullStop}}\rangle$}$\
where \TZ{T} is any token but \T{?}.
\item $\langle\text{\NT{FullStop}}\rangle = \NT{FullStop}$.
\end{itemize}
To summarize in words, after \T{?}\ must follow a defined macro name.
Either it is defined to have no parameters, in which case \T{?}\ and
the atom are replaced by the expansion and macro expansion starts over from there,
or it is defined to have a number of parameters in which case \T{?} and the atom
must be followed by that many macro arguments.  In that case
\T{?}, the atom and the macro arguments are replaced by the expansion in which each
occurrence of a macro parameter has been replaced by the corresponding macro argument
and macro expansion starts over from there.  If the first token is not \T{?}, it is
made part of the result of the macro expansion.

For example, suppose the following macro definitions are in force:
\begin{verbatim}
-define(foo,fum(?).
-define(bar(X),(X))).
\end{verbatim}
They associate the macro name \T{foo} with no parameters and the three tokens \T{fum},
\T{(} and \T{?}, and the macro name \T{bar} with one macro parameter \T{X} and an
expansion consisting of \T{(}, the token sequence given as macro argument, \T{)} and \T{)} again.
(This would be an extremely objectionable programming style.)
Then the \NT{TokenSequence}
\begin{verbatim}
foo(X) -> ?foo bar(6*X).
\end{verbatim}
which consists of the thirteen tokens \T{foo}, \T{(}, \T{X}, \T{)}, \T{->},
\T{?}, \T{foo}, \T{bar}, \T{(}, \T{6}, \T{*}, \T{X} and \T{)},
followed by a full stop, is macro expanded to
\begin{verbatim}
foo(X) -> fum((6*X)).
\end{verbatim}
in the following steps:
\begin{enumerate}
\item $\langle\text{\T{foo ( X ) -> ? foo bar ( 6 * X )} \NT{FullStop}}\rangle$ \\
The first five tokens (including the atom \T{foo} even though there happens to be a macro
with that name) are unaffected by macro expansion.
\item \T{foo ( X ) ->} $\langle\text{\T{? foo bar ( 6 * X )} \NT{FullStop}}\rangle$ \\
The next two tokens are the separator \T{?}\ and the atom \T{foo}.  There is a macro definition
of \T{foo} with no parameters and expansion consisting of the tokens \T{fum},
\T{(} and \T{?}; thus \T{?}\ and \T{foo} are replaced by \T{fum}, \T{(} and \T{?}
and macro expansion starts over from there.
\item \T{foo ( X ) ->} $\langle\text{\T{fum ( ? bar ( 6 * X )} \NT{FullStop}}\rangle$ \\
The next two tokens are unaffected by macro expansion.
\item \T{foo ( X ) -> fum (} $\langle\text{\T{? bar ( 6 * X )} \NT{FullStop}}\rangle$ \\
The next two tokens are \T{?} and \T{bar} and there is a macro definition of \T{bar} with
one macro parameter. The macro argument consists of the tokens \T{6}, \T{*} and \T{X}.
The tokens \T{?}, \T{bar}, \T{(}, \T{6}, \T{*}, \T{X} and \T{)} are then replaced with the
expansion \T{(}, \T{6}, \T{*}, \T{X}, \T{)}, \T{)} and macro expansion starts over from there.
\item \T{foo ( X ) -> fum (} $\langle\text{\T{( 6 * X ) )} \NT{FullStop}}\rangle$ \\
The remaining six tokens are unaffected by macro expansion.
\item \T{foo ( X ) -> fum ( ( 6 * X ) )} \NT{FullStop} \\
Macro expansion is complete.
\end{enumerate}

Note that directives are not macro expanded (and their arguments are not evaluated).
For example, it is not
possible to write
\begin{verbatim}
-define(SRCDIR(FN),"/usr/local/src/myproj/" FN ".erl").

-include(?SRCDIR("bliss")).
\end{verbatim}
because the argument of \T{include} must be an \NT{IncludeFileName}, i.e., a \NT{One\-String\-Literal}.
\index{macro!expansion|)}

\subsection{Initial set of macro definitions}

\label{section:initial-macros}
\index{macro!initial set|(}

Initially the following three macros
\ifOld are \fi
\ifNew must be \fi
defined, each one associated with no parameters:
\begin{itemize}
\item \T{?MODULE} is expanded to a single token: an atom literal that
is the name of the module being compiled.
\item \T{?FILE} is expanded to a single token which is a string
literal that is a full path to the file that is being compiled.
\item \T{?LINE} is expanded to a single token which is an integer
literal that is the number of the line on which the \T{LINE} token
appears.  If \T{LINE} occurs within a macro then \T{LINE} is expanded
to the number of the line in which that macro is expanded.  Undefining
or redefining the \T{LINE} macro has no effect. (Note that \T{LINE}
does not have a fixed token sequence and has to be handled as a
special case in the macro expansion.)
\ifOld
\item \T{?MACHINE} is expanded to a single token: an atom literal that
is the name of the abstract \Erlang\ machine on which the compiler is
running. The known abstract \Erlang\ machines are:
\begin{itemize}
\item \T{'JAM'}
\item \T{'BEAM'}
\item \T{'VEE'}
\end{itemize}
% Atoms for new abstract machines should be registered with Ericsson Software Technology AB.
\fi
\iffalse
\item \T{?VERSION} is expanded to an integer token. An implementation supporting
\NewErlang\ should expand \T{?VERSION} to \T{500}.  An implementation supporting
\OldErlang\ should expand \T{?VERSION} to \T{470}.
% Hellre lista???
\fi
% ?DATE ???
\end{itemize}
\ifNew
An implementation may provide additional predefined macros, which
should be documented.
\fi
\index{macro!initial set|)}

\section{File inclusion}

\label{section:include}
\index{include directive@\T{include} directive|(}
\index{include_lib directive@\T{include_lib} directive|(}
\index{file!inclusion of|(}

The \T{include} and \T{include_lib} directives splice in the contents
of a file.

\begin{rules}
\grrule{IncludeDirective}
       {\TXT{-} \TXT{include} \TXT{(} \NT{IncludeFileName} \TXT{)} \NT{FullStop}}

\grrule{IncludeLibDirective}
       {\TXT{-} \TXT{include_lib} \TXT{(} \NT{IncludeFileName} \TXT{)} \NT{FullStop}}

\grrule{IncludeFileName}
       {\NT{OneStringLiteral}}
\end{rules}

The difference between the \T{include} and \T{include_lib} directives
lies in which paths are searched when the given filename is not
absolute.  For \T{-include(\Z{F})}, file \TZ{F} is searched in each
directory for which an include path was given to the compiler (option
\T{\{i,\Z{Dir}\}}), in order (as if using \T{file:path_open/3} \cite[p.~230]{otp-dev-ref}).
Also for \T{-include_lib(\Z{F})}, these paths are tried first.
However, if the file is not found, then if \TZ{F} can be split into a
string \TZ{L} that does not contain the character `\T{/}' and one
string \TZ{N} that begins with `\T{/}', \TZ{L} is looked up as a
library (i.e., as if it was given as argument to \T{code:lib_dir/1}
\cite[p.~164]{otp-dev-ref}).  If that yields a path \TZ{P} to a
library directory then the path which is the concatenation of
\TZ{P} and \TZ{N} is used.

It is a compile-time error if no readable file is found.

The contents of the file is subject to lexical processing
(\S\ref{chapter:lexical}).  It is an error if lexical processing of
the contents of the included file does not yield a
\NT{TerminatedTokens}, i.e., a sequence of \NT{TokenSequence}.

The resulting sequence of \NT{TokenSequence} is processed exactly as
described in this section.  While processing the included file, the
\T{FILE} and \T{LINE} macros (\S\ref{section:initial-macros}) refer to
the path of the included file and line numbers in it, respectively.
The sequence of \NT{TokenSequence} that is the result of the
preprocessing of the included file becomes part of the output of the
preprocessing of the including sequence of \NT{TokenSequence} at the
location of the \T{include} or \T{include_lib} directive.
\index{include directive@\T{include} directive|)}
\index{include_lib directive@\T{include_lib} directive|)}
\index{file!inclusion of|)}

\section{Conditional compilation}

\label{section:condcomp}
\index{conditional compilation|(}
\index{ifdef directive@\T{ifdef} directive|(}
\index{ifndef directive@\T{ifndef} directive|(}
\index{else directive@\T{else} directive|(}
\index{endif directive@\T{endif} directive|(}

There are four directives related to conditional compilation and two additional
directives for which syntax is reserved.

\begin{rules}
\grrule{IfdefDirective}
       {\TXT{-} \TXT{ifdef} \TXT{(} \NT{MacroName} \TXT{)} \NT{FullStop}}
\grrule{IfndefDirective}
       {\TXT{-} \TXT{ifndef} \TXT{(} \NT{MacroName} \TXT{)} \NT{FullStop}}
\grrule{ElseDirective}
       {\TXT{-} \TXT{else} \NT{FullStop}}
\grrule{EndifDirective}
       {\TXT{-} \TXT{endif} \NT{FullStop}}
\end{rules}

It is a compile-time error if the sequence of \NT{TokenSequence} given
to the preprocessing step does not contain matching triples or pairs
of
\begin{itemize}
\item \NT{IfdefDirective}, \NT{ElseDirective} and \NT{EndifDirective};
\item \NT{IfndefDirective}, \NT{ElseDirective} and \NT{EndifDirective};
\item \NT{IfdefDirective} and \NT{EndifDirective};
\item \NT{IfndefDirective} and \NT{EndifDirective}.
\end{itemize}
This must hold for a whole module definition as well as individually for each
included file (\S\ref{section:include}).

\index{skipping mode|(}
\index{processing mode|(}
If the compiler is in processing mode and
encounters an \T{-ifdef(\Z{M})} directive [or \T{-ifndef(\Z{M})} directive],
the compiler tests whether there is a
macro definition for \TZ{M}.  If this is [not] the case, then the compiler continues in
processing mode until it encounters the matching \T{else} or \T{endif} directive.  (Note that this
may involve handling any number of enclosed nested triples or pairs of the kinds above.)
If an \T{else} directive was encountered, the compiler continues in skipping [processing] mode until the
matching \T{endif} directive is encountered.  In either case, after the \T{endif} directive,
the compiler continues in processing mode.

If the compiler is in skipping mode and encounters an \T{-ifdef(\Z{M})}
or \T{-ifndef(\Z{M})} directive, it continues in skipping mode until
it encounters the matching \T{else} or \T{endif} directive.  (Again,
this may involve handling any number of enclosed nested triples or
pairs of the kinds above.)  If an \T{else} directive was encountered,
the compiler continues in skipping mode until the matching \T{endif}
directive is encountered.  In either case, after the \T{endif}
directive, the compiler continues in skipping mode.  Thus, it is
obligatory for the compiler to keep track of the directives for
conditional compilation also when in skipping mode.  For example, the
compiler must report a compile-time error for a module containing the
following directives:
\begin{verbatim}
-define(foo,42).
-ifdef(foo).
-else.
-else
-endif.
\end{verbatim}

The compiler may implement this mechanism in any suitable way, but the
following machinery serves as an example and may clarify the
description above.  Let the compiler have a state consisting of a
register \T{Processing} and a stack.  \T{Processing} is
either \T{true} or \T{false}.  Each stack item is a pair where the
left half is either \T{if} or \T{else} and the right half is either
\T{true} or \T{false}. Initially \T{Processing} is \T{true} and the
stack is empty.  Here is what happens when the compiler encounters a
\NT{TokenSequence}:
\begin{itemize}
\item If the \NT{TokenSequence} is \T{-ifdef(\Z{M})} or \T{-ifndef(\Z{M})}, then
a pair of \T{if} and the value of \T{Processing} is pushed onto the stack.
If \T{Processing} is \T{true} and either
the \NT{TokenSequence} is \T{-ifdef(\Z{M})} and \TZ{M} has no macro definition,
or the \NT{TokenSequence} is \T{-ifndef(\Z{M})} and \TZ{M} has a macro definition,
then \T{Processing} is changed to \T{false}.
\item If the \NT{TokenSequence} is \T{-else} then it is a compile-time error if
the stack is empty or the left half of the top pair is not \T{if}.  The left half
of the top pair is changed to \T{else} and the contents of \T{Processing} is set
to the logical conjunction of the right half of the top pair and
the negation of \T{Processing}.
\item If the \NT{TokenSequence} is \T{-endif} then it is a compile-time error if
the stack is empty.  \T{Processing} is set to the right half of the top stack pair
and that pair is popped off the stack.
\item If the \NT{TokenSequence} is anything else, then if \T{Processing} is \T{true},
it is processed as described in \S\ref{section:preprocessing}, otherwise it is
ignored.
\end{itemize}
\index{skipping mode|)}
\index{processing mode|)}

The directives \T{if}\index{if@\T{if}!(reserved) directive}
and \T{elif}\index{elif (reserved) directive@\T{elif} (reserved) directive}
(with any number of arguments) are
reserved for future extension of conditional compilation.
\index{conditional compilation|)}
\index{ifdef directive@\T{ifdef} directive|)}
\index{ifndef directive@\T{ifndef} directive|)}
\index{else directive@\T{else} directive|)}
\index{endif directive@\T{endif} directive|)}
\index{module!preprocessing of|)}

\section{Parsing}

\label{section:parsing}
\index{module!parsing|(}

The sequence of \NT{TokenSequence} is parsed as a
\NT{ModuleDeclaration} (\S\ref{section:module-declarations}).
\index{parse tree!result of parsing|(}
The result is a list of parse trees
(one for each top-level form), which are
represented as \Erlang\ terms (\S\ref{chapter:parse-trees}).
\index{parse tree!result of parsing|)}
\index{module!parsing|)}

\section{Parse transforms}

\label{section:parse-transform}
\index{parse transform|(}
\index{parse tree!in parse transform|(}

A parse transform is a function mapping lists of parse trees to lists of
parse trees.  Each parse transform is implemented by a separate module
with an exported function named
\T{parse_transform/1}\index{parse_transform/1 function@\T{parse_transform/1} function}.
Suppose that the compiler has been instructed (through compiler options)
to use the $k$
parse transforms implemented by modules $\TZ{M}_1$, \ldots, $\TZ{M}_k$.
(The order is relevant and is the same as the order in which the compiler
options appear.)

Let $\TZ{L}_0$ be the list of parse trees resulting from parsing the
preprocessed sequence
of \NT{TokenSequence}, represented as \Erlang\ terms (\S\ref{chapter:parse-trees}).
For each $i$, $1\leq i\leq k$, let $\TZ{L}_i$ be the result of computing
\begin{alltt}
\(\Z{M}\sb{i}\):parse_transform(\(\Z{L}\sb{i-1}\))
\end{alltt}
if that computation completes normally.  The result must be a list of parse
trees for a \NT{ModuleDeclaration} (just as $\TZ{L}_0$ is).
If for some $\TZ{M}_i$, the computation completes abruptly or the result of the
computation is not a list of parse trees for a \NT{ModuleDeclaration}, then
it is a compile-time error.
Otherwise, $\TZ{L}_k$ is used in the code
generation step (\S\ref{section:code-generation}).
\index{parse transform|)}
\index{parse tree!in parse transform|)}

\section{Code generation}

\label{section:code-generation}
\index{module!code generation|(}

\index{parse tree!input to code generation|(}
\index{binary!representation of compiled module|(}
The code generation step takes a list of parse trees represented as \Erlang\ terms
(\S\ref{chapter:parse-trees}) and produces a binary that contains a loadable
representation of the module.
\index{parse tree!input to code generation|)}
That binary can either be loaded immediately
(\S\ref{section:loading}) or be written to a file for later loading.
The actual format of the binary is implementation-defined.
\index{binary!representation of compiled module|)}
\index{module!code generation|)}

\iffalse
% MID stuff
\section{Loading a module}

\label{section:loading}

A binary is either available immediately, for example, directly from code generation
(\S\ref{section:loading}) or read from a file.

Loading the binary results in a \emph{module term}, as 
described in \S\ref{section:module-terms}.  It is also possible to update
the mapping of module names to module terms on the node onto which the binary is loaded,
as described in \S\ref{section:module-names}.
\fi
